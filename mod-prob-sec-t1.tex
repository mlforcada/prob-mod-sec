\documentclass{beamer}
%\documentclass[handout]{beamer}
 \mode<presentation>{ \usetheme{Warsaw}  \setbeamercovered{transparent}
   %\usecolortheme{albatross} % for inverted color
 \usecolortheme{wolverine} % way too flashy!
    %\usefonttheme{structurebold}
    %\usetheme{Berkeley}
    % or ...    \setbeamercovered{transparent}
    % or whatever (possibly just delete it)
 
 }
%\mode<handout>

\usepackage[english]{babel}
\usepackage[utf8]{inputenc}
\usepackage{times}
\usepackage{url}
\usepackage[T1]{fontenc}
\newcommand{\Pair}[2]{\texttt{#1}\(\leftrightarrow\)\texttt{#2}}
\newcommand{\pair}[2]{\texttt{#1}\(\to\)\texttt{#2}}

\title[Desambiguación con MOM]{Desambiguación léxica categorial usando modelos  usando modelos ocultos de Markov}

\author[Mikel  L.\ Forcada]{Mikel L.\ Forcada\inst{1}}

\institute[Universitat d'Alacant]{ 
\inst{1}Departament de Llenguatges i Sistemes Informàtics,
Universitat d'Alacant,  E-03071 Alacant (Spain) 
}


\date[2015--2014]{Doctorado en Informática}



\newcommand{\tu}[2]{(\texttt{``#1''},\texttt{``#2''})}
\begin{document}

% twiddling with the colors
%\setbeamercolor{title}{fg=red!80!black} title red
%\setbeamercolor{title}{bg=green} % title background green

% \setbeamercolor{alerted text}{fg=darkblue!80!yellow}
\setbeamercolor*{palette primary}{fg=red!60!black,bg=yellow!80!green}
%\setbeamercolor*{palette secondary}{fg=red!70!black,bg=yellow!90!green}
%\setbeamercolor*{palette tertiary}{bg=red!80!black,fg=yellow!50!green}
\setbeamercolor*{palette quaternary}{fg=red,bg=yellow!60!green}

% \setbeamercolor*{sidebar}{fg=darkblue,bg=orange!75!white}

% \setbeamercolor*{palette sidebar primary}{fg=darkblue!10!black}
% \setbeamercolor*{palette sidebar secondary}{fg=white}
% \setbeamercolor*{palette sidebar tertiary}{fg=darkblue!50!black}
% \setbeamercolor*{palette sidebar quaternary}{fg=yellow!10!orange}

% \setbeamercolor*{titlelike}{parent=palette primary}
% \setbeamercolor{frametitle}{bg=yellow!90!green}
\setbeamercolor{frametitle right}{bg=yellow!50!green}

% \setbeamercolor*{separation line}{}
% \setbeamercolor*{fine separation line}{}


\frame{\maketitle}
% \begin{frame}
% \maketitle
% \titlepage
% \end{frame}

\begin{frame}<beamer>
\frametitle{Índice}
\tableofcontents
\end{frame}

\section{Definiciones}
\begin{frame}
\frametitle{Etiquetario}

\(\Gamma\): etiquetario o conjunto de etiquetas (categorías
  lexicas, partes de la oración, etc., más finas o más gruesas).

\[\Gamma=\{\gamma_1,\gamma_2,\ldots,\gamma_{|\Gamma|}\}\]

\end{frame}
%%%%%%%%%%%%%%%%%%%

\begin{frame}
\frametitle{ Observables}

{

{\(\Sigma\): conjunto de \emph{observables} (palabras concretas, clases
  de ambigüedad ---conjuntos de las etiquetas asignadas a cada palabra--- o
  cualquier representación entre estas)} 

{\[\Sigma=\{\sigma_1,\sigma_2,\ldots,\sigma_{|\Sigma|}\}\]}

 {Tras un cierto procesamiento, el texto es una secuencia de observables.}


}
\end{frame}

%%%%%%%%%%%%%%%%%%%%%%%%%%%%%%%%

\begin{frame}
\frametitle{ Conjunto de etiquetas}

{
{\(T:\Sigma\to 2^\Gamma\) entrega el conjunto de etiquetas \(T(\sigma)\) de un
  observable \(\sigma\).}

{Si el observable es una clase de ambigüedad, \(T\) es la función
  identidad.}

}
\end{frame}
%%%%%%%%%%%%%%%%%%%%%%%%%%%%%%%%%%%%%
\section{Formulación probabilística de la desambiguación}
\begin{frame}
\frametitle{ El problema}

{
{Tenemos un texto procesado (una secuencia de observables de longitud \( L \))

 \[\mathbf{\sigma}=\sigma[1]\sigma[2]\ldots\sigma[L]\]

}

{y queremos conocer qué secuencia de etiquetas

 \[\mathbf{\gamma}^*=\gamma^*[1]\gamma^*[2]\ldots\gamma^*[L]\]

es la que mejor explica el texto.}

}
\end{frame}
%%%%%%%%%%%%%%%%%%%%%%%%%%%%%%%%%%%%%
\begin{frame}
\frametitle{Formulación probabilística/1}

{
{Si pudiéramos calcular la probabilidad

 \[P(\gamma[1]\gamma[2]\ldots\gamma[L]|\sigma[1]\sigma[2]\ldots\sigma[L])\]
 
 de cada secuencia de etiquetas \(\gamma[1]\gamma[2]\ldots\gamma[L]\)
 condicionada a la secuencia de observables
 \(\sigma[1]\sigma[2]\ldots\sigma[L]\), }

{y quisiéramos conocer qué secuencia de etiquetas
 \(\mathbf{\gamma}^*=\gamma^*[1]\gamma^*[2]\ldots\gamma^*[L]\)
es la que mejor explica el texto, podríamos elegir la más probable:

\[\gamma^*[1]\ldots\gamma^*[L]=\mathrm{arg}
 \max_{\gamma[i]\in T(\sigma[i])} P(\gamma[1]\ldots\gamma[L]|\sigma[1]\ldots\sigma[L]).\]
}

}
\end{frame}


%%%%%%%%%%%%%%%%%%%%%%%%%%%%%%%%%%%%%
\begin{frame}
\frametitle{ Formulación probabilística/2}

{
{Aplicando el teorema de Bayes,
{\small
\[
 P(\gamma[1]\ldots\gamma[L]|\sigma[1]\ldots\sigma[L])=
\frac{ P(\sigma[1]\ldots\sigma[L]|\gamma[1]\ldots\gamma[L])
 P(\gamma[1]\ldots\gamma[L])} {P(\sigma[1]\ldots\sigma[L])}
,\]
}
}

{y, teniendo en cuenta que para un texto dado,
  \(P(\sigma[1]\ldots\sigma[L])\) es constante, podemos escribir:
{\small
\[\gamma^*[1]\ldots\gamma^*[L]=\mathsf{arg}
 \max_{\gamma[i]\in T(\sigma[i])} P(\sigma[1]\ldots\sigma[L]|\gamma[1]\ldots\gamma[L])
 P(\gamma[1]\ldots\gamma[L])\]
}
}
}
\end{frame}
%%%%%%%%%%%%%%%%%%%%%%%%%%%%%%%%%%%%%
\begin{frame}
\frametitle{ Formulación probabilística/3}


Tenemos, por tanto, dos modelos probabilísticos:
\begin{itemize}
\item Un \emph{modelo léxico} 
\[ P(\sigma[1]\ldots\sigma[L]|\gamma[1]\ldots\gamma[L]) \]
que indica la probabilidad de una secuencia de observables dada una
secuencia de etiquetas

\item y un \emph{modelo sintáctico}
\[P(\gamma[1]\ldots\gamma[L])\]
que indica la probabilidad de una determinada secuencia de etiquetas
\end{itemize}


\end{frame}

%%%%%%%%%%%%%%%%%%%
%%%%%%%%%%%%%%%%%%%%%%%%%%%%%%%%%%%%%
\begin{frame}
\frametitle{ Aproximaciones}

{
{\emph{Modelo léxico:} La probabilidad de cada observable sólo depende de la etiqueta correspondiente:
\[P(\sigma[1]\ldots\sigma[L]|\gamma[1]\ldots\gamma[L])
\approx \prod_{t=1}^L p(\sigma[t]|\gamma[t]) \]
}
{\emph{Modelo sintáctico:} La probabilidad de una secuencia de etiquetas se puede aproximar por un modelo de bigramas:

\[P(\gamma[1]\ldots\gamma[L])\approx p(\gamma[1]|\#)\left(\prod_{t=2}^L
p(\gamma[t]|\gamma[t-1])\right) p(\#|\gamma[L]).\]}
}
\end{frame}
%%%%%%%%%%%%%%%%%%%

%%%%%%%%%%%%%%%%%%%%%%%%%%%%%%%%%%%%%
\section{Modelos ocultos de Markov per a la desambiguación}
\begin{frame}
\frametitle{ Modelo oculto de Markov}

{
{Un modelo oculto de Markov \(\mathcal{M}=(A,B,\pi)\) se compone de:}

\begin{itemize}\itemsep 0ex
{\item \emph{Probabilidades de transición} (matriz \(A\))
\[a_{ij}=p(\gamma_j|\gamma_i)\]
donde $\gamma_j$ puede ser también \(\#\) (final del texto).}

{\item \emph{Probabilidades de emisión} (matriz \(B\))
\[b_{i}(\sigma_k)=p(\sigma_k|\gamma_i)\]}

{\item \emph{Probabilidades iniciales} (vector \(\mathbf{\pi}\))
\[a_{\#i}=p(\gamma_i|\#)=\pi_i\]
}
\end{itemize}
}
\end{frame}
%%%%%%%%%%%%%%%%%%%%%%%%%%%%%%%%%%%%%%%%%%%%%%%%%%%%%%%%%%%%%%%%%%%%
\section{El algoritmo de Viterbi}
\begin{frame}
\frametitle{ El algoritmo de Viterbi/1}

{
{
La manera obvia (fuerza bruta) de obtener la secuencia óptima 
  \[\gamma^*[1]\ldots\gamma^*[L]\] (calcular la probabilidad de  todas
  las secuencias \(\gamma[1]\ldots\gamma[L]\) hasta que encontremos la más probable) es impracticable.}

{Existe un algoritmo (el de Viterbi) que la calcula
  eficientmente.} 

{Lo repasaremos sin justificarlo detalladamente.}

}
\end{frame}
%%%%%%%%%%%%%%%%%%%%%%%%%%%%%%%%%%%%%%%%%%%%%%%%%%%%%%%%%%%%%%%%%%%%
\begin{frame}
\frametitle{El algoritmo de Viterbi/2}

{
{El algoritmo calcula, procesando los observables \(\sigma[t]\)
  uno a uno,}
\begin{itemize}
{\item una \emph{probabilidad} \(\delta_i(t)\) para cada
  \(\gamma_i\in\Gamma\), y }
{\item una \emph{función de traza} \(\psi(i,t)\) para  cada
  etiqueta \(\gamma_i\) y para cada longitud \(t\) que
  permitirá obtener la secuencia ganadora.}
\end{itemize}
}
\end{frame}
%%%%%%%%%%%%%%%%%%%%%%%%%%%%%%%%%%%%%%%%%%%%%%%%%%%%%%%%%%%%%%%%%%%%
\begin{frame}
\frametitle{ Etiquetando con Viterbi/1
}
{
{\emph{Iniciación:}
\[\delta_i(0)=\pi_i=a_{\#i}\]
}
{\emph{Paso a paso, de \(1\) a \(L\)}
\[\delta_i(t)= \max_{\gamma_j\in\Gamma} \delta_j(t-1) a_{ji} b_i(\sigma[t]) \]
\[\psi(i,t)= \mathrm{arg} \max_{j\in[1,|\Gamma|]} \delta_j(t-1) a_{ji} b_i(\sigma[t]) \]
}
{\emph{Terminación}
\[\delta_\#(L+1)= \max_{\gamma_j\in\Gamma} \delta_j(L) a_{j\#} \]
\[\psi(\#,L+1)= \mathrm{arg} \max_{j\in[1,|\Gamma|]} \delta_j(L) a_{j\#} \]
}

}
\end{frame}
%%%%%%%%%%%%%%%%%%%%%%%%%%%%%%%%%%%%%%%%%%%%%%%%%%%%%%%%%%%%%%%%%%%%
\begin{frame}
\frametitle{ Etiquetando con Viterbi/2}

{
{La secuencia de etiquetas ganadora se calcula hacia atrás:
\[\gamma^*[L]=\gamma_{\psi(\#,L+1)},\]
\[\gamma^*[L-1]=\gamma_{\psi(\psi(\#,L+1),L)},\]
\[\gamma^*[L-2]=\gamma_{\psi(\psi(\psi(\#,L+1),L),L-1)},\]
etc.
}

}
\end{frame}
%%%%%%%%%%%%%%%%%%%%%%%%%%%%%%%%%%%%%%%%%%%%%%%%%%%%%%%%%%%%%%%%%%%%
\begin{frame}
\frametitle{ Etiquetando con Viterbi/3}

{
{Algunas propiedades interesantes:}
\begin{itemize}\itemsep 0ex
{\item Los \(b_i(\sigma_k)\) tienen muchos \emph{ceros}:
       \[ b_i(\sigma_k)=0 \mbox{ si } \gamma_i\not\in T(\sigma_k) \]
}
{\item La subsecuencia ganadora \(\gamma^*[t_i+1]...\gamma^*[t_f-1]
\) entre dos palabras no homógrafas \(\sigma[t_i]\) i \(\sigma[t_f]\) (tales
que \(|T(\sigma[t_i])|=|T(\sigma[t_f])|=1\)) es independiente del resto de la secuencia: ello permite \emph{restaurar} el algoritmo y \emph{escribir}
la desambiguación antes de llegar al final del texto.}

% {\item 
% Si \(|T(\sigma[t_i])|=|T(\sigma[t_f])|=1\) amb \(t_i<t_f\) (mots no
% homògrafs), la subseqüència guanyadora \(\gamma^*[t_i+1]...\gamma[t_f-1]^*
% \) és independent de la resta
% de la seqüència: permet \emph{restaurar} l'algorisme i \emph{escriure}
% la desambiguació abans d'arribar al final del text.}
\end{itemize}
}
\end{frame}
\section{Estimación de los parámetros}
\begin{frame}
\frametitle{ Estimación de los parámetros/1}

{
{Pero... de dónde vienen los parámetros (\(A\),\(B\),\(\mathbf{\pi}\))?}

{Si tenemos un corpus suficientemente grande y representativo  etiquetado (desambiguado) a mano, 
conocemos \(\sigma[t]\) i \(\gamma^*[t]\) per a tot \(t\). }

{Podemos estimar las probabilidades \emph{contando} (verosimilitud màxima):
\[ a_{ij}\approx \frac{N(\gamma^*[t]=\gamma_j|\gamma^*[t-1]=\gamma_i)}
                     {\sum_{\gamma_k\in\Gamma} N(\gamma^*[t]=\gamma_k|\gamma^*[t-1]=\gamma_i)}  \]
\[ b_{i}(\sigma_k)  \approx
\frac{N(\sigma[t]=\sigma_k|\gamma^*[t]=\gamma_i)}
     {\sum_{\sigma_l\in\Sigma} N(\sigma[t]=\sigma_l|\gamma^*[t]=\gamma_i)} \]         
}
}
\end{frame}
%%%%%%%%%%%%%%%%%%%%%%%%%%%%%%%%%%%%%%%
\begin{frame}
\frametitle{ Estimación de los parámetros/2}

{
{Puede convenir tener en cuenta la posibilidad de situaciones no vistas (por ejemplo, asumiendo que las hemos visto al menos una vez):}

{\[ a_{ij}\approx \frac{1+ N(\gamma^*[t]=\gamma_j|\gamma^*[t-1]=\gamma_i)}
                     {\sum_{\gamma_k\in\Gamma} \left(1+
                     N(\gamma^*[t]=\gamma_k|\gamma^*[t-1]=\gamma_i)\right)
                     }\]}


{\[ b_{i}(\sigma_k)  \approx
\frac{1+N(\sigma[t]=\sigma_k|\gamma^*[t]=\gamma_i)}
     {\sum_{\sigma_l\in\Sigma} \left(1+ 
N(\sigma[t]=\sigma_l|\gamma^*[t]=\gamma_i) \right)} \]}


}
\end{frame}
%%%%%%%%%%%%%%%%%%%%%%%%%%%%%%%%%%%%%%%
\begin{frame}
\frametitle{ Estimación de los parámetros/3}

{
{Pero tener un corpus etiquetado comporta \emph{un gran esfuerzo}.}

{Existe un algoritmo (el de Baum y Welch), también denominado de
  \emph{maximitzación de la esperanza}, que permite estimar sin necesidad de un corpus etiquetat, mediante un proceso iterativo.}

{No lo explicaremos aquí. Pero indicaremos una manera
  \emph{aproximada} de obtener los parámetros (por ejemplo, para
  refinarlos posteriormente con el algoritmo de Baum y Welch).}
}
\end{frame}

%%%%%%%%%%%%%%%%%%%%%%%%%%%%%%%%%%%%%%%
\begin{frame}
\frametitle{ Estimación de los parámetros/4}

{
{\emph{Model sintàctic:} Si el corpus no está etiquetado, no
  conocemos los
  \(N(\gamma^*[t]=\gamma_j|\gamma^*[t-1]=\gamma_i)\), pero podemos
  aproximarlos:}
{
\[
\tilde{N}(\gamma^*[t]=\gamma_j|\gamma^*[t-1]=\gamma_i)\approx
\]
\[
\approx \sum_{\sigma:\gamma_j\in\sigma} \sum_{\sigma':\gamma_i\in\sigma'}
N(\sigma[t]=\sigma|\sigma[t-1]=\sigma') \frac{1}{|T(\sigma)|} \frac{1}{|T(\sigma')|}
\]
}
{(cada vez que aparece el observable \(\sigma\) que tiene
  \(|T(\sigma)|\) etiquetas posibles, contamos como si hubiéramos visto
  cada etiqueta \(\frac{1}{|T(\sigma)|}\) veces)}
}
\end{frame}
%%%%%%%%%%%%%%%%%%%%%%%%%%%%%%%%%%%%%%%
\begin{frame}
\frametitle{ Estimación de los parámetros/5}

{
{\emph{Modelo léxico:} Si el corpus no está etiquetado, no conocemos
los  \(N(\sigma[t]=\sigma_k|\gamma^*[t]=\gamma_i)\), pero podemos aproximarlos:}
{
\[
\tilde{N}(\sigma[t]=\sigma_k|\gamma^*[t-1]=\gamma_i)\approx
N(\sigma[t]=\sigma_k) \frac{1}{|T(\sigma_k)|}
\]
}
{(Igual: cada vez que aparece el observable \(\sigma\) que tiene
  \(|T(\sigma)|\) etiquetas posibles, contamos como si hubiéramos visto cada etiqueta \(\frac{1}{|T(\sigma)|}\) veces)}
}
\end{frame}

%%%%%%%%%%%%%%%%%%%%%%%%%%%%%%%%%%%%%%%
\section{Uso de información lingüística}
\begin{frame}
\frametitle{ Información lingüística}

{
{Cuando se trabaja con corpus no etiquetados puede ser conveniente
  usar \emph{información lingüística} para hacer que algunas probabilidades de transición \(a_{ij}\) sean cero.}

{Por ejemplo, en español las secuencias
  \emph{determinante}--\emph{verbo en forma personal},
  \emph{preposición}--\emph{verbo en forma personal}, \emph{pronombre
    proclítico}--\emph{adjectivo} no son posibles.}

{El algoritmo de Baum y Welch \emph{conserva} durante las iteraciones los ceros que se hayan introducido en la inicialización.}
}
\end{frame}

\end{document}
